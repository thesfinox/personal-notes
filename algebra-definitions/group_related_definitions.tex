\documentclass[10pt,a4paper]{article}

% PDF version and output
\pdfoutput=1        % force the execution of pdflatex
\pdfminorversion=7  % newest PDF version (1.7)

% Packages
\usepackage[utf8]{inputenc}
\usepackage{amsmath}
\usepackage{amsfonts}
\usepackage{amssymb}
\usepackage{geometry}
\usepackage{dsfont}
\usepackage{enumerate}
\usepackage{bookmark}   % loads hyperref

\geometry{
    a4paper,
    left=30mm,
    right=30mm,
    top=40mm,
    bottom=40mm,
}

% Informations
\author{Riccardo Finotello}
\title{Algebra Basic Definitions}
\date{}

% Hyperref informations
\hypersetup
{
    pdftitle={Basic Definitions},
    pdfsubject={Lie Groups},
    pdfauthor={Riccardo Finotello},
    pdfkeywords={lie groups, algebra, mathematics, definitions, geometry},
    pdfstartview={FitH},
    pdfcreator={pdfTeX TeX Live 2019/Arch Linux},
    pdfproducer={pdfLaTeX},
    pdflang={en-GB},
    pdfpagemode={UseOutlines},
    bookmarksopen={true},
    bookmarksnumbered={true},
    hidelinks
}

\renewcommand\labelenumii{\theenumi\alph{enumii}.}

\begin{document}
    \pagenumbering{gobble}

    \maketitle

    \begin{itemize}
        \item \textbf{Semigroup} $\left( \mathrm{G}, \cdot \right)$
            $\longrightarrow$ e.g.: $\left( \mathds{N}\: \backslash
            \left\lbrace 0 \right\rbrace,\ + \right)$, $\left( \mathds{Z},\
            \mathrm{min} \right)$:
            \begin{enumerate}[i)]
                \item $\mathrm{G}$ is a \textit{set},
                \item $\cdot \colon \mathrm{G} \times \mathrm{G} \rightarrow
                    \mathrm{G},\; \left( g, h \right) \mapsto l\; \mid\;
                    \left( g \cdot h \right) \cdot l = g \cdot \left( h \cdot l
                    \right)\;\;\; \forall g,h,l \in \mathrm{G}$.
            \end{enumerate}

        \item \textbf{Monoid} $\left( \mathrm{M}, \cdot \right)$
            $\longrightarrow$ e.g.: $\left( \mathds{N},\ + \right), \left(
            \mathds{N}\: \backslash \left\lbrace 0 \right\rbrace,\ \cdot
            \right)$:
            \begin{enumerate}[i)]
                \item $\mathrm{M}$ is a \textit{semigroup},
                \item $\exists\, \mathds{1}_M\; \mid\; m \cdot \mathds{1}_M =
                    \mathds{1}_M \cdot m = m\;\;\; \forall m \in \mathrm{M}$.
            \end{enumerate}

        \item \textbf{Group} $\left( \mathds{G}, \cdot \right)$
            $\longrightarrow$ e.g.: $\left( \mathds{Z}_n,\ \cdot \right)$,
            $\left( \mathrm{S}_n,\ \cdot \right)$, $\left( \mathrm{SO}\left( n
            \right),\ \cdot \right)$:
            \begin{enumerate}[i)]
                \item $\mathds{G}$ is a \textit{monoid},
                \item $\forall g \in \mathds{G}\; \exists\, g^{-1} \in
                    \mathds{G}\; \mid\; g \cdot g^{-1} = g^{-1} \cdot g =
                    \mathds{1}_{\mathds{G}}$.
            \end{enumerate}

        \item \textbf{Ring} $\left( \mathds{R},\ +,\ \cdot \right)$
            $\longrightarrow$ e.g.: $\left( \mathds{Z},\ +,\ \cdot \right)$,
            $\mathrm{End}\left( \mathds{G} \right)$ where $\mathds{G}$ is an
            \textit{abelian group}:
            \begin{enumerate}[i)]
                \item $\left( \mathds{R},\ + \right)$ is an \textit{abelian
                    group} (i.e. $g + h = h + g\;\;\; \forall g,h \in
                    \mathds{R}$),
                \item $\left( \mathds{R},\ \cdot \right)$ is a \textit{monoid},
                \item $r \cdot \left( s + t \right) = r \cdot s + r \cdot
                    t\;\;\; \forall r,s,t \in \mathds{R}$,
                \item $\left( r + s \right) \cdot t = r \cdot t + s \cdot
                    t\;\;\; \forall r,s,t \in \mathds{R}$.
            \end{enumerate}

        \item \textbf{Division ring} $\left( \mathrm{R},\ +,\ \cdot \right)$
            $\longrightarrow$ e.g.: $\left( \mathds{H},\ +,\ \cdot \right)$:
            \begin{enumerate}[i)]
                \item $\left( \mathrm{R},\ +,\ \cdot \right)$ is a \textit{ring},
                \item $\forall r \in \mathrm{R}\; \exists\, r^{-1} \in
                    \mathrm{R}\; \mid\; r \cdot r^{-1} = r^{-1} \cdot r =
                    \mathds{1}_{\mathrm{R}}$.
            \end{enumerate}

        \item \textbf{Field} $\left( \mathds{F},\ +,\ \cdot \right)$
            $\longrightarrow$ e.g.: $\left( \mathds{C},\ +,\ \cdot \right)$:
            \begin{enumerate}[i)]
                \item $\left( \mathds{F},\ +,\ \cdot \right)$ is a
                    \textit{commutative ring} (i.e. $f \cdot g = g \cdot
                    f\;\;\; \forall f,g \in \mathds{F}$),
                \item $\exists\, \mathds{1}_{\mathds{F}} \in \mathds{F}\;
                    \mid\; \mathds{1}_{\mathds{F}} \cdot f = f \cdot
                    \mathds{1}_{\mathds{F}} = f\;\;\; \forall f \in \mathds{F}$.
            \end{enumerate}

        \item $\mathrm{R}$\textbf{-module} $\mathrm{M}$ $\longrightarrow$ e.g.:
            $\mathrm{C}^{\infty}\left( \mathrm{M} \right)$-module of $X \in
            \Gamma\left( \mathrm{M},\ \mathrm{TM} \right)$:
            \begin{enumerate}[i)]
                \item $\left( \mathrm{M},\ + \right)$ is an \textit{abelian
                    group},
                \item $\left( \mathrm{R},\ +,\ \cdot \right)$ is a
                    \textit{ring},
                \item $\cdot \colon \mathrm{R} \times \mathrm{M} \rightarrow
                    \mathrm{M},\; \left( r, m \right) \mapsto r \cdot m\; \mid
                    \; r \cdot \left( m + n \right) = r \cdot m + r \cdot
                    n\;\;\; \forall r \in \mathrm{R},\; \forall m,n \in
                    \mathrm{M}$,
                \item $\cdot \colon \mathrm{R} \times \mathrm{M} \rightarrow
                    \mathrm{M},\; \left( r, m \right) \mapsto r \cdot m\; \mid
                    \; \left( r + s \right) \cdot m = r \cdot m + s
                    \cdot m\;\;\; \forall r,s \in \mathrm{R},\; \forall m \in
                    \mathrm{M}$,
                \item $\cdot \colon \mathrm{R} \times \mathrm{M} \rightarrow
                    \mathrm{M},\; \left( r, m \right) \mapsto r \cdot m\; \mid
                    \; \left( r \cdot s \right) \cdot m = r \cdot \left( s
                    \cdot m \right)\;\;\; \forall r,s \in \mathrm{R},\; \forall
                    m \in \mathrm{M}$,
                \item $\cdot \colon \mathrm{R} \times \mathrm{M} \rightarrow
                    \mathrm{M},\; \left( r, m \right) \mapsto r \cdot m\; \mid
                    \; \mathds{1}_{\mathrm{R}} \cdot m = m \cdot
                    \mathds{1}_{\mathrm{R}} = m\; \forall m \in \mathrm{M}$.
            \end{enumerate}

        \item $\mathds{F}$\textbf{-vector space} $\mathrm{V}$ $\longrightarrow$
            e.g.: $\left( \mathds{R}^n,\ +,\ \cdot \right)$
            \begin{enumerate}[i)]
                \item $\mathrm{V}$ is a $\mathds{F}$\textit{-module},
                \item $\mathds{F}$ is a \textit{field}.
            \end{enumerate}
    \end{itemize}
\end{document}
