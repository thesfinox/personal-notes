\documentclass[10pt,a4paper]{article}

% PDF version and output
\pdfoutput=1        % force the execution of pdflatex
\pdfminorversion=7  % newest PDF version (1.7)

% Packages
\usepackage[utf8]{inputenc}
\usepackage{amsmath}
\usepackage{amsfonts}
\usepackage{amssymb}
\usepackage{graphicx}
\usepackage{booktabs}   % \toprule, \midrule, \bottomerule
\usepackage{tikz}
\usetikzlibrary{decorations.pathmorphing,decorations.pathreplacing,
decorations.markings,patterns}
%\usepackage{setspace}   % \begin{spacing} to change line spacing (titlepage)
%\usepackage{bold-extra} % use \textbf and \textsc together
%\usepackage{authblk}    % multiple authors
\usepackage{fancyhdr}   % fancy layout
\usepackage{bookmark}   % loads hyperref

% Pagestyle (fancyhdr package)
\pagestyle{fancy}
\fancyhead[L]{\footnotesize Riccardo Finotello}
\fancyhead[R]{\footnotesize \rightmark}
\fancyfoot[C]{\thepage}
\renewcommand{\headrulewidth}{1pt}

% New numbering style (numbering up to subsections for equations and figures)
\numberwithin{equation}{subsection}
\numberwithin{figure}{subsection}
\numberwithin{table}{subsection}

% Debug packages
%\usepackage{refcheck}              % shows eq and fig labels on PDF
%\usepackage[showframe]{geometry}   % shows page frame (overfull \hbox debug)

% Informations
\author{Riccardo Finotello}
\title{Stress Energy Tensor - OPE}
\date{\today}

% Hyperref informations
\hypersetup
{
    pdftitle={Stress Energy Tensor - OPE},
    pdfsubject={Conformal Field Theory},
    pdfauthor={Riccardo Finotello},
    pdfkeywords={cft,stress tensor,ope,virasoro,algebra},
    pdfstartview={FitH},
    pdfcreator={pdfTeX Tex Live 2015/Debian},
    pdfproducer={pdfLaTeX},
    pdflang={en-GB},
    pdfpagemode={UseOutlines},
    bookmarks={true},
    bookmarksopen={true},
    bookmarksnumbered={true},
    hidelinks
}

\pagenumbering{gobble}

\begin{document}

    \maketitle

    Consider the Virasoro algebra:
    \begin{equation*}
        \left[ L_m,\, L_n \right] = \left( m - n \right) L_{m+n} + \frac{c}{12}
        m \left( m^2 - 1 \right) \delta_{m+n,\,0},
    \end{equation*}
    where $\delta_{i,j}$ is the usual Kronecker delta. This is encoded in
    \begin{equation*}
        T\left( z \right) T\left( w \right) = \frac{c/2}{\left( z - w
        \right)^4} + \frac{2}{\left( z - w \right)^2} T\left( w \right) +
        \frac{1}{z - w} \partial_w T\left( w \right).
    \end{equation*}

    As a matter of fact, we have:
    \begin{align*}
        \left[ L_m,\, L_n \right] &= \oint_{0,\, \left| z \right| > \left| w
        \right|} \frac{\mathrm{d}z}{2\pi i} \oint_0 \frac{\mathrm{d}w}{2\pi i}
        z^{m+1} w^{n+1} T\left( z \right) T\left( w \right) -
        \\
        &- \frac{\mathrm{d}z}{2\pi i} \oint_0 \frac{\mathrm{d}w}{2\pi i}
        \oint_{0,\, \left| z \right| < \left| w \right|} z^{m+1} w^{n+1}
        T\left( w \right) T\left( z \right) =
        \\
        &= \oint_0 \frac{\mathrm{d}w}{2\pi i} w^{n+1} \oint_w
        \frac{\mathrm{d}z}{2\pi i} z^{m+1} \mathrm{R}\left( T\left( w \right)
        T\left( z \right) \right) =
        \\
        &= \oint_0 \frac{\mathrm{d}w}{2\pi i} w^{n+1} \oint_w
        \frac{\mathrm{d}z}{2\pi i} z^{m+1} \left( \frac{c/2}{\left( z - w
        \right)^4} + \frac{2}{\left( z - w \right)^2} T\left( w \right) +
        \frac{1}{z - w} \partial_w T\left( w \right) \right) =
        \\
        &= \oint_0 \frac{\mathrm{d}w}{2\pi i} w^{n+1} \left( \frac{1}{3!}
        \frac{c}{2} \left( m + 1 \right) m \left( m - 1 \right) w^{m-2} + 2
        T\left( w \right) \left( m + 1 \right) w^m + w^{m+1} \partial_w T\left(
        w \right) \right) =
        \\
        &= \oint_0 \frac{\mathrm{d}w}{2\pi i} \left( \frac{c}{12} m \left( m^2
        - 1 \right) w^{n+m-1} + 2 \left( m+1 \right) T\left( w \right)
        w^{n+m+1} + w^{n+m+2} \partial_w T\left( w \right) \right) =
        \\
        &= \frac{c}{12} m \left( m^2 - 1 \right) \delta_{m+n,\,0} + 2 \left( m
        + 1 \right) L_{m+n} - \oint_0 \frac{\mathrm{d}w}{2\pi i} \left( n + m +
        2 \right) w^{n+m+1} T\left( w \right) =
        \\
        &= \frac{c}{12} m \left( m^2 - 1 \right) \delta_{m+n,\,0} + 2 \left( m
        + 1 \right) L_{m+n} - \left( n + m + 2 \right) L_{m+n} =
        \\
        &= \left( m - n \right) L_{m+n} + \frac{c}{12} m \left( m^2 - 1 \right)
        \delta_{m+n,\,0}.
    \end{align*}

\end{document}

